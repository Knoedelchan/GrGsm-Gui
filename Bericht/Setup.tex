%###########################################################################
%
%   Kapitel 2
%
%###########################################################################
\chapter{Setup des Raspberry PI}
Im Folgenden wird beschrieben wie wir den Raspberry aufgesetzt haben und welche Probleme uns dabei begegnet sind.
\section{Das How To und interdependencies}

gr-gsm wird hauptsächlich von Piotr Krysik entwickelt und wird von diesem auch gepflegt. Es gibt auch einen vorgeschlagenen Weg wie man gr-gsm auf einem Raspberry Pi 3 zu installieren hat. Dieser Weg hat sich allerdings als Sackgasse erwiesen. Grund dafür war, dass in der Zwischenzeit GNU Radio weiterentwickelt wurde und gr-gsm sich dieser Weiterentwicklung angepasst hatte. Die Weiterentwicklung betraf leider nicht Raspbian Jessie, was zu folgenden Problemen führte.

gr-gsm benötigte damals eine Version von GNU Radio > 3.7.9, für Raspbian stand allerdings nur die Version 3.7.6 zur Verfügung. Der Versuch auf instabile Testversionen 3.7.10+ auszuweichen erwies sich ebenso als aussichtslos. Das Installieren von diesen Versionen führte nur dazu, dass wiederum Bibliotheken von denen GNU Radio abhängig war inkompatibel wurden.

Nach einiger Recherche wie man ein funktionierendes Gesamtpaket erhalten könnte sind wir auf das Programm PyBombs gestoßen. Dieses macht genau das, was wir gesucht haben, es installiert ein Programm und alle Abhängigkeiten die dieses zur Nutzung benötigt. Im Gegensatz zu apt-get, wie man es von Ubuntu und sonstigen Linux Distributionen kennt, bezieht sich PyBOMBS nicht nur auf die Standard Repositorys und die dort hinterlegte Softwareversionen sondern installiert genau die Versionen die benötigt werden. Hierbei kann sich PyBOMBS auf eigens hinterlegte Repositorys, sogenannte Recipes, beziehen von wo es die Software aus den Quelldateien kompiliert. 
So schön es klingt führte es zu den selben Problemen wie im ersten Versuch, nur dass dieses mal andere Programme inkompatibel wurden.


An diesem Punkt haben wir beschlossen, dass eine zeitnahe Umsetzung mit gr-gsm wohl aussichtslos erscheint. Gleichzeitig standen wir mit Piotr Krysik in Kontakt um einen Weg zu finden wie es doch zu lösen sein könnte. Seither ist Herr Krysik dabei seine Anpassungen in einem Raspberry Emulator zu überprüfen. Wir entschieden uns dennoch Airprobe zu versuchen, da es nicht absehbar war wann wir mit einer erfolgversprechenden Antwort hätten rechnen können. Dafür haben wir uns an einer jahrealten Anleitung orientiert und sind dabei auf Kali Linux gestoßen, dieses war glücklicherweise auch für Raspberrys verfügbar. Nachdem wir das installiert hatten, war GNU Radio mittlerweile zu Neu um mit Airprobe arbeiten zu können. Allerdings war die stabile Version des Standard Repsitory glücklicherweise 3.7.10. Somit war das installierte GNU Radio mit dem neusten gr-gsm kompatibel und so konnten wir alle Abhängigkeiten installieren, die Kalibrierung durchführen und gr-gsm installieren. Zu der Zeit haben wir schon einen Monat Arbeit in das Projekt stecken müssen.

Im Endeffekt ist die Anleitung wie sie in Kapitel \ref{Install} dokumentiert wird, unter Kali Linux voll durchführbar. Es wäre allerdings wünschenswert den GSM Scanner unter Raspbian zum laufen zu bringen, da es sich hierbei um ein besser auf den Raspberry zugeschnittenes Betriebssystem handelt. 
Damit diese Portierung auf Raspbian möglich ist, müssen die vorher beschriebenen Probleme mit den Abhängigkeiten gelöst werden. 

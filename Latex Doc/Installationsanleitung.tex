\lstnewenvironment{code}[1][]%
{
   \noindent
   \minipage{\linewidth} 
   \vspace{0.5\baselineskip}
   \lstset{basicstyle=\ttfamily\footnotesize,frame=single,#1,breaklines=true}}
{\endminipage}

\chapter{Installationsanleitung}

Ausführlichere Anleitung und weitere Wiki-Einträge sind hier nachzulesen:\\
\url{https://github.com/ptrkrysik/gr-gsm/wiki/Installation-on-RaspberryPi-3}


\section{Installation von benötigter Software für gr-gsm}

\subsection{Installieren von Kalibrate}

Als erstes installieren wir Kalibrate:

\begin{code}[	numbers=left,stepnumber=1]
sudo apt-get install libtool autoconf automake libfftw3-dev librtlsdr0 librtlsdr-dev libusb-1.0-0 libusb-1.0-0-dev
git clone https://github.com/asdil12/kalibrate-rtl.git
cd kalibrate-rtl
git checkout arm_memory
./bootstrap
./configure
make
sudo make install 
\end{code}

\subsection{Zugriff auf USB-Device freischalten}

Das RTL-SDR Gerät einstecken und ID mit dem Befehl \colorbox{lightgray}{\emph{lsusb}} überprüfen. Zu sehen sollte etwas wie folgend sein:

\begin{code}
Bus 001 Device 004: ID **0bda:2832** Realtek Semiconductor Corp. RTL2832U DVB-T
Bus 001 Device 003: ID 0424:ec00 Standard Microsystems Corp. SMSC9512/9514 Fast Ethernet Adapter
Bus 001 Device 002: ID 0424:9514 Standard Microsystems Corp.
Bus 001 Device 001: ID 1d6b:0002 Linux Foundation 2.0 root hub
\end{code}

\noindent{}\\In unserem Fall ist die ID \textbf{0bda:2832}. Anschließend öffnen wir eine rules-Datei:

\begin{code}
sudo nano /etc/udev/rules.d/20.rtlsdr.rules
\end{code}

\noindent\\...in welcher dann folgende Zeile hinzugefügt werden muss:

\begin{code}
SUBSYSTEM=="usb", ATTRS{idVendor}=="0bda", ATTRS{idProduct}=="2832", GROUP="adm", MODE="0666", SYMLINK+="rtl_sdr"
\end{code}

\noindent\\Falls mehrere RTL-SDR Geräte verwendet werden, können mehrere Zeilen hinzugefügt werden. Die ID muss jeweils natürlich entsprechend des Gerätes abgewandelt werden.\\
\\Danach sollte der Raspberry Pi neugestartet werden: \colorbox{lightgray}{sudo reboot}

\subsection{Kalibrierung des RTL-SDR Gerätes}
Jetzt können wir den Befehl ausführen um das RTL-SDR Gerät zu kalibrieren (um genau zu sein um den durschnittlichen absoluten Fehler in ppm zu berechnen):

\begin{code}
kal -s GSM900 
\end{code}

\noindent\\Das Ergebnis sollte ähnlich zu diesem sein:

\begin{code}
Found 1 device(s):
  0:  Generic RTL2832U

Using device 0: Generic RTL2832U
Found Rafael Micro R820T tuner
Exact sample rate is: 270833.002142 Hz
kal: Scanning for GSM-900 base stations.
GSM-900:
	chan: 1 (935.2MHz - 33.430kHz)	power: 55085.23
	chan: 3 (935.6MHz - 34.130kHz)	power: 63242.36
	chan: 5 (936.0MHz - 33.970kHz)	power: 41270.82
...
...
	chan: 112 (957.4MHz - 32.934kHz)	power: 498930.07
	chan: 116 (958.2MHz - 31.859kHz)	power: 88039.44
	chan: 124 (959.8MHz - 32.429kHz)	power: 247404.23
\end{code}

\noindent\\Das stärkste Signal wäre in diesem Fall Kanal 112. Also führen wir die Kalibrierung auf diesem Kanal durch:

\begin{code}
kal -c 112 
\end{code}

\noindent\\...und erhalten ein Ergebnis wie folgt:

\begin{code}
Found 1 device(s):
  0:  Generic RTL2832U

Using device 0: Generic RTL2832U
Found Rafael Micro R820T tuner
Exact sample rate is: 270833.002142 Hz
kal: Calculating clock frequency offset.
Using GSM-900 channel 112 (957.4MHz)
average		[min, max]	(range, stddev)
- 34.368kHz		[-34376, -34357]	(20, 4.697051)
overruns: 0
not found: 0
average absolute error: 35.897 ppm
\end{code}

\noindent\\Unser durschschnittlicher absoluter Fehler wäre hier also 36 ppm (35.897 ppm).

\subsection{Installation von GNU Radio}

Als nächstes installieren wir GNU Radio:

\begin{code}
sudo apt-get install gnuradio gnuradio-dev
\end{code}

\subsection{Installieren von libosmocore}

Libosmocore muss kompiliert werden...

\begin{code}[numbers=left, stepnumber=1]
sudo apt-get install cmake
sudo apt-get install build-essential libtool shtool autoconf automake git-core pkg-config make gcc
sudo apt-get install libpcsclite-dev libtalloc-dev
git clone git://git.osmocom.org/libosmocore.git
cd libosmocore/
autoreconf -i
./configure
make
sudo make install
sudo ldconfig -i
cd
\end{code}

\noindent\\...außerdem benötigen wir noch ein paar andere Dinge.

\begin{code}[numbers=left, stepnumber=1]
sudo apt-get install gr-osmosdr rtl-sdr
sudo apt-get install libboost-dev
sudo apt-get install osmo-sdr libosmosdr-dev
sudo apt-get install libusb-1.0.0 libusb-dev
sudo apt-get install libboost-all-dev libcppunit-dev swig doxygen liblog4cpp5-dev python-scipys
\end{code}

\section{Installation von gr-gsm}

Und nun zum letzten Schritt:

\begin{code}[numbers=left, stepnumber=1]
git clone https://github.com/ptrkrysik/gr-gsm.git
cd gr-gsm
mkdir build
cd build
cmake ..
make
sudo make install
sudo ldconfig
\end{code}

\noindent\\Zuletzt erstellen wir noch die ~/.gnuradio/config.conf config-Datei, mit \colorbox{lightgray}{\emph{nano ~/.gnuradio/config.conf}}. Und fügen diese zwei Zeilen hinzu (damit GNU Radio die custom Blöcke von gr-gsm finden kann):

\begin{code}
[grc]
local_blocks_path=/usr/local/share/gnuradio/grc/blocks
\end{code}
